\begin{document}
Aircraft manufacturing and air traffic control are among the most stringently regulated industries in the world. We take the issue of losing aircraft over oceans extremely seriously. Locating a plane of any size in any of the vast oceans of our planet is very challenging. Even when debris from a downed aircraft has been located, finding where the rest of that airplane has come to rest remains a daunting task. While we stand behind the integrity of all aircraft industries, we must be prepared to conduct oceanic searches for missing flights. As a preemptive measure, we have developed a comprehensive search scheme that is built on decades of mathematical and statistical theory and that can be broadly implemented in any of the traveled regions of the world’s expansive seas.
Our search scheme consists of a two-step model: an aerial phase, followed by an underwater search. The aerial phase of the search scans the surface of the ocean for debris. In determine from where the debris may have drifted, our model draws on existing oceanic current models for the particular body of water that is being searched. Information on currents also allows for tracking the plane’s path from the surface to the seafloor. In this way, we are able to determine the best places to initially deploy underwater search vessels. Using well established search theory based on statistical reasoning, our model then intelligently guides the search vessels to the subsequently “next best” areas to search.
A notable property of our search scheme is that it can be adapted to many different settings. Our model has the capacity to handle any type of technology, any number of search vessels, and any size of search area. The effectiveness of the technology used in searching that particular area, and, as previously mentioned, the particular oceanic currents of the particular area being searched allow us to facilitate an optimal search for any aircraft that has been lost over the open ocean. 
	The loss of life that comes inherent in the disappearance of manned aircraft is truly tragic. More important than the actual location of the plane is the possible rescue of individuals in the crash or the ability to provide closure to family members of those lost. While the aeronautical industries of the world remain vigilant to prevent the disappearance of manned aircraft from happening again, our search scheme provides an effective plan to deal with such a situation should it arise.
\end{document}