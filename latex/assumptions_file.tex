\section{Assumptions}

Assumptions may be broken into three categories: general assumptions, assumptions used for the aerial search, and assumptions used for the underwater search. A note that an assumption ``is extensible'' is used to indicate that the removal of the assumption from the model does not significantly complicate the mathematics used.

\subsection{General Assumptions}

\begin{itemize}
\item Only one search occurs in a cell at a given time. This assumption is extensible.
\item When computing distances, each cell size is small enough to approximate Euclidean distance.
\item Search success can be measured directly as a probability of success, regardless of surroundings and extenuating conditions.
\item It is assumed that the time between issuing a search instruction and the time until that search begins is approximated by the distance the vessel has to move.
\end{itemize}

\subsection{Aerial Search Assumptions}

\begin{itemize}
\item An accurate model regarding ocean currents' ability to shift objects exists that is suitable for use in conjunction with this model. Such a current model would need to be able to advance forward in continuous-time or small discrete-time steps, as the probability distribution of debris will change vastly both over time and number of searches. Previous research suggests such a model exists.
\item Debris is either present or not present. Various types of debris and their individual implications are not considered. This assumption is extensible, though it would require significant modification to do so.
\end{itemize}

\subsection{Underwater Search Assumptions}

\begin{itemize}
\item Unlimited time is allowed for the underwater search. If limited time is allotted for an underwater search, then this would change the search regime slightly. This modification is not included in this paper, and constitutes a noticeable mathematical change.
\item A current model exists that can predict by probability distribution the position of the plane body on the sea floor.
\item While the plane may drift from the time of impact to when it hits the sea floor, once the plane is in contact with the sea floor, it does not move. 
\item Ocean floor depth is not relevant to the detection of the craft. While this is likely a weak assumption, it \textit{is} extensible. Based on the technical specifications of a given craft, a probability of success could be defined for a specific technology based on various impacting factors. For instance, $\beta_t(j,k)$ could be defined as a function of average floor depth at cell $j$.
\end{itemize}