\documentclass[a4paper]{article}
\usepackage[english]{babel}
%\usepackage[utf8x]{inputenc}
\usepackage{amsmath}
\usepackage{amssymb}
\usepackage{graphicx}
\usepackage{multicol}
\usepackage{textcomp}
\usepackage{gensymb}
\usepackage{listings}
\usepackage{color}
\usepackage[margin=1in]{geometry}
\usepackage{fancyhdr}

\title{Debris-Informed Underwater Search and Rescue Optimized in Two Parts}
\author{}
\date{}

\newcommand*{\Scale}[2][4]{\scalebox{#1}{$#2$}}
\pagestyle{fancy}
\fancyhf{}
\lhead{Team 40670}
\rhead{Page \thepage\, of ***}

\definecolor{dkgreen}{rgb}{0,0.6,0}
\definecolor{gray}{rgb}{0.5,0.5,0.5}
\definecolor{mauve}{rgb}{0.58,0,0.82}
\definecolor{black}{rgb}{0,0,0}

\renewcommand\thesection{\arabic{section}}
\renewcommand\thesubsection{\thesection.\arabic{subsection}}
\renewcommand\thesubsubsection{\thesubsection.\arabic{subsubsection}}

%\setlength\parindent{0em}
%\setlength{\parskip}{5pt}
%\renewcommand{\theenumi}{\Alph{enumi}}
\begin{document}\maketitle

\begin{abstract}
A model for optimal discrete-effort search and rescue for a plane in in water is presented. The search is split into two parts: the first is an aerial search for debris, and the second is an underwater search for the remainder of the plane body. Both of these are generalized with respect to the search technology used. A current model (not included in this paper) is used to generate probability distributions for debris and the plane body in the search area. This transformation allows debris data to aid the search for a lost plane. Searches in the water are then optimized based on the likelihood of discovery versus the cost of search. Both the likelihood of discovery and cost may be changed on a per-technology basis, allowing for the simultaneous use of any number of different search technologies. Results of test runs demonstrate a logarithmic relationship between percent success, number of search vehicles, and number of searches. A direct linear relationship was found between cost of searching, number of search vehicles, and number of searches. The model does not function optimally when cost is factored in. Computation time is linear with respect to iterations and number of ships. 
\end{abstract}

\pagebreak 

\section{Introduction}

The model is partitioned into two separate searches. First, a visual aerial search is performed for floating debris. Locations and times of discovery of debris are indexed. 

An existing model for the oceanic currents of the region is presumed to already exist. This model is then applied to convert positional data of debris into a probable area of the plane crash.  This positional prediction data is combined with data from the ocean current model in order to create a more localized search area of the wreckage on the sea floor. A second underwater search, informed by the aerial search, is then conducted via underwater vehicles utilizing sonar. The underwater search maximizes search effectiveness while minimizing cost and distance of searching. 

\section{General Discrete-effort Optimal Searching}

The discrete-effort method of Lawrence Stone's \textit{Theory of Optimal Search} is used in both searches to prioritize and distribute search efforts. An overview of this method is provided here. 

\subsection{Defining the Terms Used in Searching}

The search space $\mathcal{J}$ is divided into an integer number of uniformly square cells $j\in\mathcal{J}$. The specific technologies (i.e. one type of Autonomous Underwater Vehicle versus another) used in the search are indexed for convenience by integers in $\mathcal{T}$. This allows for implementation of different measurement accuracies and cost functions as search technology advances, or simply as different technology is chosen. 

For each search technology $t\in\mathcal{T}$ used, each cell $j\in\mathcal{J}$ is associated with a cost of searching. This cost is expressed in terms of the cell number and the number of times searched: $\gamma_t(j,k)$. This represents the cost of the $k$th search in cell $j$ using technology $t$.

It is important to note that $\gamma_t(j,k)$ can be defined with respect to a general cost function, $c_t(j,k)$; by contrast, $c_t$ is the cost of searching  a \textit{total} of $k$ discrete times. Ergo, $$\gamma_t(j,k_n)=c_t(j,k_n)-c_t(j,k_{n-1})$$

The probability of debris existing in a given cell is denoted by $p(j)$.

Finally, $\beta_t(j,k)$ is the probability of locating an object in cell $j$ using tech $t$ on the $k$th attempt, assuming that the object is actually in cell $j$.

The quantity $p(j)\beta(j,k)$ represents the actual chance of finding debris in the $j$th cell on the $k^{th}$ search. $\beta$ assumes the object is in the current cell being searched, where the product $\beta * p$ does not assume so.

Next, the optimization problem is stated. The optimal individual cell to focus the search methods on is the one which provides the highest chance of success per unit search cost. This value, denoted as $\epsilon$, is defined as $$\epsilon_t\equiv\epsilon_t(j,k)=\frac{p(j)\beta_t(j,k)}{\gamma_t(j,k)}$$. 

An $\epsilon$ can be computed for each cell, accounting for the number of searches $k$ for each one. The highest values are the ones with the greatest probability of success per unit cost. 

\subsection{Assigning Values to $\beta_t(j,k)$}

Suppose $\alpha_t$ is the success rate for locating an object, whether underwater or aerial, with $t\in\mathcal{T}$. Then, from \textit{Theory of Optimal Search}: \[\beta_t(j,k)=\alpha_t\cdot(1-\alpha_t)^{k-1}\]

This representation follows that each successive search has an exponentially decreasing chance of success. The derivation is not copied here. It is worth noting, however, that the $\alpha_t$ value will significantly depend on the technology used and the context within the model is applied.

\subsection{Cost Functions as Prioritization}

The cost $\gamma_t(j,k)$ provides flexibility for the model in terms of prioritization. Considerations such as distance, time, and money may be confounding variables in the cost of searching. The resulting function $\gamma$ will depend entirely on the technology being used. In this model, the cost of searching a cell is assumed to be equal to the distance traveled to reach the cell. In this way, distance, travel time, and monetary cost are roughly minimized.

In this model, the distance parameter in the cost of the function, such that $\gamma_t(j,k)=d_{travel}$, where $d_{travel}$ is the Euclidean distance in cells from the current position to the position under consideration. This has the benefit of moving ships around as little as possible. One may extend the cost function by including the time to search - in other words, $\gamma(j,k)=t_{search}+d_{travel}$. The inclusion of travel time implies specifying the technology to be used, however; as a result, this method of prioritization is not included here.  

\subsection{Evolving Probability Distributions over Failed Searches}

In the event that the model does not find an object in cell $k$, the probability distribution must be modified to accommodate for the altered search area. Let $T$ represent the probability that the plane is in the cell, and $F$ the probability that it is found. Then: 
$$P(T|F)=\frac{P(T)P(F|T)}{P(F)}$$ 
$$P(T|\sim F)=\frac{p(j)[1-\beta_t(j,k)]}{(p(j)[1-\beta_t(j,k)]+(1-p(j))}$$

By reduction, the posterior probability that the object is there but was not found is: $$p'(j)=p(j)\frac{1-\beta_t(j,k)}{1-\beta_t(j,k)p(j)},\;\mbox{where }j\mbox{ is searched}$$.

The posterior probability of every other square can be similarly adjusted by the knowledge that an object was not discovered in cell $k$ (though its proof is omitted): $$p'(j)=p(j)\frac{1}{1-p(j)\beta_t(j,k)},\;\mbox{where }j\mbox{ is not searched.}$$

After the application of these transformations, as $p'$ is a probability distribution, the quantity $$\sum_{j\in\mathcal{J}}p'(j) = 1$$

(The notation here may be confusing. For clarification, the first $p'(j)$ adjustment applies only to the square that was searched; the second applies to every other square.)

After a failed search, the probability of the plane being in cell $j$ becomes this probability.

\subsection{Executing the Search Method}

Suppose there is a set $\mathcal{T}$ of search technologies and $n_t$ instances of technology $t\in\mathcal{T}$. For each device, the $\epsilon$ value is computed. Optimizing this value returns the highest probability of object discovery per unit cost.

Each device is allocated sequentially to a square that is a) not already being searched, and b) has the highest $\epsilon$ value for that technology. Should two devices conflict in their optimal square, the closest device of that type claims it. Though travel time is not modeled, it is reasonable to make preliminary decisions in deference to it.

Whenever a device finishes searching, the probability distribution of the object's position is recomputed according to the above transformation, the device's $\epsilon$ value is recomputed based on the new distribution, and it is assigned to a new cell.

\section{Assumptions}

Assumptions may be broken into three categories: general assumptions, assumptions used for the aerial search, and assumptions used for the underwater search. A note that an assumption ``is extensible'' is used to indicate that the removal of the assumption from the model does not significantly complicate the mathematics used.

While many of these assumptions can be removed from the model, the actual process for doing so is typically not included in this paper; however, an overview is given for all extensible assumptions.

\subsection{General Assumptions}

\begin{itemize}
\item A known search space already exists between departure point and destination point.
\item Only one search occurs in a cell at a time; only one vessel occupies a cell at a given time. This assumption is extensible by programmatically allowing multiple vessels to be in a given cell. However, the process for computing the probability of success is complicated by this extension, and is excluded from this paper.
\item When computing distances, the curvature of the Earth is ignored; each cell size is considered small enough to approximate Euclidean distance. This is done to simplify the consideration of a two-dimensional grid; extension is fully possible through the inclusion of a concrete distance.
  
\item It is assumed that the time between issuing a search instruction and the time until that search begins is approximated by the distance the vessel has to move. This assumption is only partially extensible; inclusion of time into the model requires reworking the $\gamma$ function to vary with both distance and time.
\item All cells are of equal size. This is relatively extensible. In order to adjust for this, however, one would need to adjust the probability of success, the cost of searching, and the probability of the object being in the cell according to the area of the cell. This is not a simple process, and it is not included here.
\end{itemize}

\subsection{Aerial Search Assumptions}

\begin{itemize}
\item An accurate model regarding ocean currents' ability to shift objects exists that is suitable for use in conjunction with this model. Such a current model needs to advance in either continuous-time or small discrete-time steps, as the probability distribution of debris will change vastly both over time and number of searches. Previous research suggests such a model exists. 
\item Debris is either present or not present. Various types of debris and their individual implications are not considered. 
\end{itemize}

\subsection{Underwater Search Assumptions}

\begin{itemize}
\item Unlimited iterations of searching are allowed. While it would be possible to allow for finite searching without changing the algorithm, the search algorithm, as-is, would no longer be optimal. 
After all available effort has been allocated, the algorithm would need to provide a best-guess answer. Hence, a constraint on effort allocation places an additional constraint on optimization.  This modification constitutes a significant mathematical change.
\item A current model exists that can predict an initial probability distribution of the position of the plane body on the sea floor. This probability distribution will be later informed by discovered debris.
\item The plane does not drift once it makes contact with the sea floor.  
\item The success of a vehicle searching a space can be measured directly as a probability of success, regardless of surroundings and extenuating conditions. This assumption also implies that either the sea floor is of uniform depth and topography, or that the depth and topography of the sea floor is not relevant to plane discovery. While this is likely a weak assumption, this \textit{it} extensible simply by making the $\beta_t(j,k)$ function vary with cell $j$ as a function of sea floor depth. 
\end{itemize}

\section{Current Flow Modeling and Aerial Search}

Existing ocean current models are sufficient for determining approximately where a discovered piece of debris could have come from. These current models are not included in this paper, although their outputs significant. and are declared as follows:

A current model is assumed to generate two probability distributions: the first is the distribution of debris, and the second is the positional distribution of the plane. The debris distribution is $p_d(j)$ (for $j\in\mathcal{J}$). The plane's positional distribution is considered naive at the outset, as it is not yet informed by the position of debris. The naive positional distribution is $p_n(j)$. 

Additionally, it is expected that the current model be able to adjust the positional probability of the plane based on the originating point of the debris. This is expressed with the operator $\mathtt{T}$. This will be defined later. 

The first step in this model for object search is to perform an aerial debris search. 

Planes will search through cells in a manner specified by the general optimal search model. The highest discovery rate to cost ratio will be investigated first, then each other cell sequentially.

The presence of debris at cell $j$ is $d_j\in \{0,1\}$. It is assumed that there either is or is not debris in a cell. If debris is present, then $\mathtt{T}$ modifies the probability distribution $p_n$ in order to represent the likelihood that debris came from a plane crash. This is expressed as: $$\mathtt{T}(d_j,p_n)=p_n'$$

Repeated application of $\mathtt{T}$ over each cell $d_j$ will create the new positional distribution informed by debris, $p_w$. This is created by: $$p_w=\mathtt{T}(d_1, \mathtt{T}(d_2, \ldots \mathtt{T}(d_j, p_n)\ldots))$$

\section{Underwater Plane Body Search}
The second step in the model is performing the underwater search. 

Autonomous underwater vehicles will follow the general search method previously stated in section 3.5. The cost function $\gamma_t(j,k)$ is slightly altered. A distance cost is added: $\gamma_t(j,k)=d$, where $d$ is the distance that would be traversed by each vessel in moving from its current location to a new target cell $k$. This cost function seeks to minimize $d$ as to overall maximize the probability of successfully locating the vessel. Analysis of the effectiveness of this function is presented in the next section.



\section{Model Analysis}

\subsection{Model Properties}

\subsection{Sensitivity Analysis}

\pagebreak

\begin{thebibliography}{10}
\bibitem{TF} Furukawa, T., Bourgault, F., Lavis, B., Durrant-Whyte F., H., \emph{``Recursive Bayesian Search-and-Tracking Using Coordinated UAVs for Lost Targets"}, IEEE International Conference on Robotics and Automation, 2006. 
\bibitem{LL} L.C.M. Lebreton, S.D. Greer, J.C. Borrero, \emph{``Numerical modelling of floating debris in the world’s oceans"}, Marine Pollution Bulletin, Volume 64, Issue 3, March 2012, Pages 653-661, ISSN 0025-326X, http://dx.doi.org/10.1016/j.marpolbul.2011.10.027.
\bibitem{Mackay98} Mackay, D.J.C, \emph{``Introduction to Monte Carlo Methods"}, submitted for publication.
\bibitem{HR} Henry R. Richardson and Lawrence D. Stone, \emph{``Operations analysis during the underwater search for Scorpion"}, submitted for publication. 
\bibitem{Stone75} Stone, Lawrence D., \textit{Theory of Optimal Search}, New York: Academic Press, 1975.
\bibitem{LS} Lawrence D. Stone, Colleen M. Keller, Thomas M. Kratzke, Johan P. Strumpfer, \emph{``Search for the Wreckage of Air France Flight AF 447"}, Statistical Science 2014, Vol. 29, No. 1, 2014.
\end{thebibliography}

\end{document}