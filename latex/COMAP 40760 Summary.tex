\documentclass[a4paper]{article}
\usepackage[english]{babel}
%\usepackage[utf8x]{inputenc}
\usepackage{amsmath}
\usepackage{amssymb}
\usepackage{graphicx}
\usepackage{multicol}
\usepackage{textcomp}
\usepackage{gensymb}
\usepackage{listings}
\usepackage{color}
\usepackage[margin=1in]{geometry}
\usepackage{fancyhdr}

\title{}
\author{}
\date{}

\newcommand*{\Scale}[2][4]{\scalebox{#1}{$#2$}}


\definecolor{dkgreen}{rgb}{0,0.6,0}
\definecolor{gray}{rgb}{0.5,0.5,0.5}
\definecolor{mauve}{rgb}{0.58,0,0.82}
\definecolor{black}{rgb}{0,0,0}

\renewcommand\thesection{\arabic{section}}
\renewcommand\thesubsection{\thesection.\arabic{subsection}}
\renewcommand\thesubsubsection{\thesubsection.\arabic{subsubsection}}
\vspace{18cm}

\begin{document}\maketitle

A model for optimal discrete-effort search and rescue for a plane in in water is presented. The search is split into two parts: the first is an aerial search for debris, and the second is an underwater search for the remainder of the plane body. Both of these are generalized with respect to the technology used. A current model (not included in this paper) is used to generate probability distributions for debris and the plane body in the search area. This transformation allows debris data to aid the search for a lost plane. Searches in the water are then optimized based on the likelihood of discovery versus the cost of search. Both the likelihood of discovery and cost may be changed on a per-technology basis, allowing for the simultaneous use of any number of different search technologies. Results of test runs demonstrate a logarithmic relationship between percent success, number of search vehicles, and number of searches. A direct linear relationship was found between cost of searching, number of search vehicles, and number of searches. The model does not function optimally when cost is factored in. Computation time is linear with respect to iterations and number of ships. 

\end{document}
